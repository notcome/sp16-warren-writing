\documentclass[12pt,letterpaper]{article}

\usepackage{ifpdf}
\usepackage{mla}

\begin{document}
\begin{mla}{Minsheng}{Liu}{Cami Koepke}{WCWP 10B}{03/31/16}{1A}

In the Introduction, Pollan asked a question: why Americans could change
their attitude towards foods overnight, redefining the standard of
healthy diet merely based on several articles (Pollan 2)? This
phenomenon, ``national eating disorder'' as he put (Pollan 2), motivated
him to trace back the whole food production chain underlying modern
foods.

In the next three chapters, he described the dominant position of corn
in our food supply system (Pollan 17) and explained how corn evolved to
such status. In chapter one, Pollan enumerated the advantages corn has
over other plants. First, unlike the normal photosynthesis cycle found
in most plants, corn has a system called ``C-4'' that is more efficient
in producing organic compound (Pollan 21). Second, corn can also adopt
different environments easily (Pollan 25). Last but not least,
scientists discovered a way to force farmers to purchase seeds of corn
every year, providing incentives for big companies to spread corns
(Pollan 31).

Then, Pollan recounted how human factors came into play. This time, his
evidence was drawn largely from historical facts as well as first-hand
report. For instance, he described how the American government decided
to turn chemical compounds for making explosives into fertilizers
(Pollan 41). By interviewing Naylor, a local farmer in Iowa state, he
revealed the shocking fact that the cost of producing a bushel of corn
is more expensive than its market price (Pollan 53). He also described
how the corn market developed an efficient system to handle such a large
amount of food, namely the concept of Number 2 corn (Pollan 59).

The conclusion Pollan drew was clear: planting corn is not good for
human beings. It has negative impact to our environment. It reduces
biodiversity and forces many species to go extinct. Eating products made
from corns, such as corn syrup, is not good for health (Pollan 62).
Neither is it good for farmers (Pollan 53, 62 -- 63). As he put in the
third chapter, corn ``could contribute to obesity and to hunger both.''
Perhaps the only beneficial part is big companies such as Cargill and
ADM (Pollan 63).

Schlosser might not like the argument made here. In his article, he
proposed that an artificial flavor is no less, if not more, healthier
than its natural counterpart (Schlosser 27). He should agree with the
idea that the industrialization has a huge impact on the food factor
though. On the contrary, Biello is likely to support Pollan's anti-corn
sentiment. In his article, Biello stated the fact that organic food with
careful farming could deliver nearly the same yield as the
``conventional methods'' (Biello 233). He also pointed out that the
pollution issue brought by the ``conventional methods'' were too large
to neglect (Biello 234). While some might say that even the industrial
way of farming could hardly eliminate starvation, Biello thought that
the real issue to solve is about food distribution and waste (Biello
234). To Biello, moving away from the corn empire would be a wise
decision.

\subsection*{Works Cited}
\bibent Pollan, Michael. \textit{The Omnivore's Dilemma: A Natural History of Four Meals}. New York: Penguin, 2006. Print.

\bibent Schlosser, Eric. ``Why the Fries Taste Good.'' \textit{Fast Food Nation}. New York: First Mariner Books, 2012. 120-129. Rpt. in \textit{Food Matters: A Bedford Spotlight Reader}. Ed. Holly Bauer. Boston: Bedford/St. Martin's, 2014. 20-29. Print.

\bibent Biello, David. ``Will Organic Food Fail to Feed the World?'' \textit{Scientific American}. Scientific American Inc., 25 Apr. 2012. Web. Rpt. in \textit{Food Matters: A Bedford Spotlight Reader}. Ed. Holly Bauer. Boston: Bedford/St. Martin's, 2014. 232-235. Print.

\end{mla}
\end{document}