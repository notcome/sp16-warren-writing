\documentclass[12pt,letterpaper]{article}

\usepackage{ifpdf}
\usepackage{mla}

\begin{document}
\begin{mla}{Minsheng}{Liu}{Cami Koepke}{WCWP 10B}{05/12/16}
  {Obesity Epidemic Cannot Be Solved Solely By Eaters}

There is an ongoing debate over what is an effective solution to the
obesity epidemic. Michale Pollan, the author of the book
\emph{Omnivore's Dilemma}, points out that highly processed
foods---especially junk foods---are the primary cause of this obesity
problem. Therefore, he argues that people should choose natural foods.
Another writer David Freedman has an entirely different view. In his
article ``How Junk Food Can End Obesity'', Freedman points out that most
people in the United States are unable to adopt natural foods as primary
sources of energy. What we should do, as Freedman proposes, is to make
processed foods less ``fattening''. Since it is impossible for most
people to change their dietary habits deliberately, I agree that the
only pragmatic solution to the obesity epidemic is to make processed
foods healthier.

Pollan believes that the processed food industry is the culprit of
America's obesity epidemic and urges people to switch to unprocessed or
even organic foods. He presents two arguments for his claim. First, the
industry relies its interests on selling more calories due to the excess
of the energy-dense corns. In the 1970s, long-running programs that
aimed at preventing overproduction were canceled in favor of a set of
cheap-food farm policy (Pollan 103). Consequently, the price for raw
foods plummeted, and the food industry became more profitable in selling
more foods (Pollan 104). Pollan gives several pieces of data to support
his argument. In particular, American's average consumption of added
sugars has grown one-fourth in the last three decades (Pollan 104).

Second, processed foods, especially fast foods, encourage people to
intake an excess of calories, which in the long run brings obesity,
diabetes, and heart disease (Pollan 117). On the one hand, processed
foods are convenient to eat. For example, Pollan recounts his family's
experience of eating McDonald's while driving (Pollan 109). He points
out that in fact nearly one-fifth of American meals are eaten in
automobiles (Pollan 110), an indication of the industry's huge success.
On the other hand, fast foods' fragrance and flavor is unique and
unforgettable, making fast foods appealing to most Americans. Indeed, as
described in ``Why the Fries Taste Good'' by Eric Schlosser, modern
flavor industry employs advanced technologies and interdisciplinary
knowledge to produce tasty additives, which makes fast foods so popular
(Schlosser 27). Fast foods provide what Pollan calls ``a signifier of
comfort food.'' Consequently, when people eat processed food, they are
likely to overeat (Pollan 119). Combining the above two aspects, one can
see that Americans' tables are dominated by a series of fat-causing
foods. It is not hard to imagine why overweight has became a major
health issue to the country.

To sum up, the food industry has made Americans fat. Moreover, because
the industry interests depend on how much Americans eat, the industry is
unlikely to save the public health spontaneously. Since the food
producers cannot be relied on to fight obesity, we consumers have to
save ourselves, which requires a switch to healthier eating. Pollan does
compose a solid argument for why the food industry causes the obesity
epidemic. However, his conclusion that it is eaters who should switch
is, as Freedman criticizes, impractical, because it is unrealistic for
the majority of Americans, both practically and mentally, to refuse
processed foods in their daily meals.

Practically speaking, most people who need to change to a healthier
dietary habit---those who are overweight and have potential health
issues---are unable to eat healthier meals due to external limitations.
As Freedman notes, there is a correspondence between the class gap and
the obesity gap: Most obese people are the working poor (Freedman 16 --
17). The first consequence of this demographic pattern is that most
obese people cannot afford unprocessed foods (Freedman 17). One can
verify this fact by comparing the price tags in Whole Foods or Trader
Joe's and menus in McDonald's. Also, as mentioned by Pollan, the
processed food industry benefits from the cheap raw materials such as
corns (Pollan 167), which makes them capable of lowering prices further
when the competition from wholesome foods appears.

What makes the situation worse is that wholesome food stores are often
absent from the neighborhoods of the working poor. Freedman points out
that stores like Whole Foods are several bus rides away (Freedman 17).
This issue is most prominent in schools. In the movie \emph{Fed Up},
several high school students shares their lunchtime experience at
school: there are no options other than those ``fattening'' junk foods
(\emph{Fed Up}, 54:00 -- 55:30). It is said that ``by 2012 more than
half of all U.S. school districts served fast foods'' (\emph{Fed Up},
56:15 -- 57:00).

Last but not least, a person from working class might not even know what
constitutes a healthy meal because that person might not know enough
about foods. In the article ``Why Shame Won't Stop Obesity,'' the doctor
Dhruv Khullar points out that even for those familiar with nutritional
knowledge, it is hard to eat healthily in a consistent way (Khullar
128), not to mention the working class. After all, as Freedman puts, an
average obese person in America is ``relatively poor, does not read
\emph{The Times} or cookbook manifestos''.

Based on all three points presented above, it is clear that it is
impractical to ask the majority of American citizens to refuse highly
processed unhealthy foods because most people lack the resources
required to embrace unprocessed foods. Nevertheless, some people might
present the counterargument that the government can subsidize the
wholesome food sector to make these foods more accessible. After all, as
Pollan notes, one reason why fast foods are so cheap is that a major raw
material for fast foods, namely corns, are highly subsidized (Pollan
107).

Then, will the obesity epidemic be eliminated when the government starts
to subsidize carrots? Freedman and others who share the same view with
him do not think so because not eating junk foods is impossible for most
people. First, one should realize that the \emph{perceived} taste of a
kind of food has a huge impact on its popularity. Freedman illustrates
the importance of taste through the example of McLean Deluxe (Freedman
26). McLean Deluxe was a product of McDonald's in the early 90s that
aimed to promote healthy eating. The product passed McDonald's taste
test, and Freedman and his wife also love it. However, the public
reaction to McLean Deluxe was purely negative, and McDonald's had to
remove this product at last. The failure was due to the fact that when
people heard that McLean Deluxe was healthy, people subconsciously
thought that it tasted worse than other products and then chose not to
eat them. Next, not only do people have a stereotype of healthy foods
that these foods taste bad, healthy foods do taste bad. One should not
forget that one reason Pollan gives about why we should turn away from
junk foods is that modern processed foods taste so well that people can
easily overeat, which I have elaborated above. Such analysis is in
harmony with our experience in the real world. In the movie \emph{Fed
Up}, a servant at the school restaurant says that although healthy foods
are available, every student chooses junk foods (\emph{Fed Up} 59:10 --
59:48).

If people cannot even stop eating unhealthy foods, how can we depend on
them---without external coercions---to solve the epidemic obesity?Some
might argue that these people do have some external coercion to force
them to eat healthily. That external force is the social pressure of
being fat. One can open YouTube and observe how ``slim'' models employed
in advertisements are. Clearly, it is a commonly accepted idea that
being fat is bad not because of potential diseases related with obesity
but purely for the sake of being fat. Nevertheless, such social
pressure, as argued by Khullar in his article ``Why Shame Won't Stop
Obesity,'' is not as effective in eliminating obesity as many have
thought. Recently, Georgia launched a campaign to increase people's
awareness of obesity. The campaign was not effective, though, because
for those who had already been overweight such a campaign, as Khullar
put, ``shifts the focus of obesity control efforts to personal
responsibility'' (Khullar 128). Consequently, social pressure cannot
change people's minds, and the only viable solution left is to improve
the junk foods.

In conclusion, Pollan's proposal that we should switch away from junk
foods cannot save the public health, for it is impossible for most
people to change. There are many reasons behind this difficulty. From a
practical view, healthy wholesome foods are both expensive and
inaccessible to the working poor, yet these working poor are those who
suffer from the obesity most. Mentally speaking, it is also hard for
people to give up those delicious fast foods and choose to stay healthy
solely by themselves. While some might think that shame might solve
obesity, in reality, it has been proven unfeasible. Hence, we should
realize that the only practical solution to the obesity epidemic is to
make processed foods healthier. It might be hard, but as argued in
Freedman's article, it is not impossible.


\newpage

\subsection*{Works Cited}
\bibent Pollan, Michael. \textit{The Omnivore's Dilemma: A Natural History of
Four Meals}. New York: Penguin, 2006. Print.

\bibent Freedman, David. ``How Junk Food Can End Obesity?" \textit{The Atlantic:
July/August 2013 issue}.  Web.

\bibent Schlosser, Eric. ``Why the Fries Taste Good."
\textit{Fast Food Nation}. New York: First Mariner Books, 2012. 120-129. Rpt.
in \textit{Food Matters: A Bedford Spotlight Reader}. Ed. Holly Bauer. Boston:
Bedford/St. Martin's, 2014. 20-29. Print.

\bibent Khullar, Dhruv. ``Why Shame Won’t Stop Obesity." \textit{Bioethics
Forum}. The Hastings Center, 28 Mar. 2012. Web. Rpt. in
\textit{Food Matters: A Bedford Spotlight Reader}.
Ed. Holly Bauer. Boston: Bedford/St. Martin’s, 2014. 127-129. Print.

\bibent Soechtig, Stephanie et. al. ``Fed Up." 2014. Documentaray

\end{mla}
\end{document}
