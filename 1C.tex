\documentclass[12pt,letterpaper]{article}

\usepackage{ifpdf}
\usepackage{mla}

\begin{document}
\begin{mla}{Minsheng}{Liu}{Cami Koepke}{WCWP 10B}{04/07/16}{1C}

People should care the fact that we are corn eaters because it has
severe negative impacts on our world. According to Pollan, industrial
food production is not sustainable and creates pollution that people
could have avoided, though the validity of such claim was questioned by
Hurst (Hurst 210). Pollen also pointed out that there are several safety
issues with industrial foods, which I have discussed in 1B and omitted
here.

Despite these two ``utilitarian'' issues with America's corn industry,
Pollan also discussed two moral issues. The first one is about the
distribution of economic gain between family farmers and big
corporations. Farmers sell corns at a price lower than the cost of
growing their corns (Pollan 53). The government subsidies corn farms to
make the system sustainable (Pollan 61), but essentially taxpayers'
money went into big corporations' pockets (Pollan 63). Clearly, Pollan
did not like the status quo and thought that those big corporations did
not \emph{deserve} the money. In other worlds, he thought the food
production sector was unfair to its labors. Another issue is regarding
the life standards of industrially grown animals, in particular cattle
fed in feedlots (Pollan 79 -- 80).

To many people, organic foods are the opposite of what I described
above. As Pollan put, organic farms should use ``a softer, more
harmonious approach'' (Pollan 143). Pollan also compared the underlying
principle between the organic approach and the conventional one (Pollan
146 -- 151). In short, while the conventional method relies on
``reductionist science'' (Pollan 150), the organic method leverages what
the nature bestows on us. Clearly, people chose the conventional
agriculture for its high immediate yields.

Nevertheless, the return of the organic agriculture is not what people
usually imagined. For instance, organic animals must have ``access to
pasture'' (Pollan 157), but in practice such access is purely symbolic
(Pollan 172). Taking a more comprehensive view of the lifecycle of an
organic food product, one could find that it is not that environmentally
friendly compared with its ``conventional'' counterpart. For every
calorie of organic salad, ``57 calories of fossil fuel energy'' is
consumed during the food storage, transport, and process. This is only 4
percent lower than conventionally grown salad (Pollan 167). It seemed
that Pollan was angry about these big organic food companies because
they tricked customers into thinking that these corporations products
are superior but the reality is the exact opposite. In other words, he
thought such business practice was immoral.

Hurst objected Pollan's viewpoints by directly commenting on the
latter's book. For instance, Hurst pointed out that chemical fertilizers
are necessary, since it is impractical to gain an equivalent amount of
nitrogen in an organic way (Hurst 210 -- 211). While what he said might
be true from a farmer's perspective, I do not think his view is
incompatible with that of Pollan. Pollan pointed out that America do not
need such a huge amount of corns (Pollan 62), and by reducing the amount
of corns required to grow, it might be possible to achieve the goal of
feeding everyone with a transition to organic agriculture. David Biello
discussed this issue in-depth in his article \emph{Will Organic Food
Fail to Feed the World?}, and the result is in favor of organic food
(Biello 233 -- 234).

Paarlberg had a stance similar to Hurst. This time, however, he
discussed the food production outside United States. He pointed to that
Africa is still suffering hunger (Paarlberg 241), and one very important
reason is that Africans use almost no fertilizers (Paarlberg 246).
Nevertheless, what he presented is not persuasive enough when one
considers data provided by Biello. Another very interesting point is
that it is technically impossible to feed animals with the organic
approach, which means that sometimes we have to make the compromise
(Paarlberg 244 -- 245). But if one can grow corns using the conventional
method \emph{without} polluting the environment, it does not sound that
bad to me.


\subsection*{Works Cited}
\bibent Pollan, Michael. \textit{The Omnivore's Dilemma: A Natural History of Four Meals}. New York: Penguin, 2006. Print.

\bibent Biello, David. ``Will Organic Food Fail to Feed the World?'' \textit{Scientific American}. Scientific American Inc., 25 Apr. 2012. Web. Rpt. in \textit{Food Matters: A Bedford Spotlight Reader}. Ed. Holly Bauer. Boston: Bedford/St. Martin's, 2014. 232-235. Print.

\bibent Paarlberg, Robert. ``Attention Whole Foods Shoppers.'' \textit{Foreign Policy}. The FP Group, May/June 2010. Web. Rpt. in \textit{Food Matters: A Bedford Spotlight Reader}. Ed. Holly Bauer. Boston: Bedford/St. Martin’s, 2014. 240-247. Print.

\bibent Hurst, Blake. ``The Omnivore’s Delusion: Against the Agri-intellectuals.'' \textit{The American}. American Enterprise Institute, 30 July 2009. Web. Rpt. in \textit{Food Matters: A Bedford Spotlight Reader}. Ed. Holly Bauer. Boston: Bedford/St. Martin’s, 2014. 204-213. Print.


\end{mla}
\end{document}
