\documentclass[12pt,letterpaper]{article}

\usepackage{ifpdf}
\usepackage{mla}

\begin{document}
\begin{mla}{Minsheng}{Liu}{Cami Koepke}{WCWP 10B}{04/02/16}{1B}

Generally speaking, there are four aspects in which the ``food system''
has something to do with our health. The first lies in the process of
producing food, such as the feedlot approach of growing cows. The second
aspect is about the pollution happened alongside the ``industrial'' food
chain. The other important issue is what people should eat given the
variety of food choices available. The last one is ``supersizing'',
which is regarded as one of the major factors of American obesity
epidemic.

Directly connected with eaters' health is the safety issue around
industrial foods. Processed foods are certainly problematic (Pollan 85
-- 99), but ``industrial cows'' are by no means safer. For instance,
because cows cannot tolerate corns, feedlots have to use antibodies to
ensure that cows could survive long enough. Consequently, superbugs
might evolve and affect the public health in a catastrophic way (Pollan
78). Another related issue is that most cows are already infected when
slaughtered. Lethal bacteria on these cows may kill beef eaters
(Pollan). USDA should have a more comprehensive research over the safety
of the beef industry. They should conduct preventive studies before it
is too late.

Pollution is another place where the food industry and our health
intersect. Pollan discussed this issue with two case studies, one about
corns (32 -- 56) and the other about feedlots (65 -- 84). For example,
the traditional grazing is ecologically sustainable---grasses are eaten
by cows and cows' manures protect the prairie; there is no pollution
(Pollan 70). By feeding cows with grains, however, people have to suffer
both the pollution of growing corns and that of cows' manures (Pollan 70
-- 71). If one considers the petroleum involved during the lifetime of a
grain-fed cow, it takes nearly a barrel of oil before one can slaughter
a cow (Pollan 84). While the meat industry enjoys the myriad profit
brought by cheap subsidized corns, it is taxpayers and the environment
that pay all the hidden costs (Pollan 82 -- 83). The government should
consider regulation in agriculture products as well as enforce stricter
environment protection laws.

The third health issue affecting most Americans is what to eat. There
are so many conflict food suggestions available that people do not know
who to believe. Even though USDA did have its guideline (USDA 112 --
113), people seem not to trust the agency (Nestle 72). The situation is
caused by the ineffectiveness of nutrition science research, as noted by
Marion Nestle in his essay \emph{Eating Made Simple}. As Nestle
explained, the current approaches employed in the field have two main
problems (73). First, it is the ``overall dietary pattern'' rather than
some isolated variable in one's meal that affects people's health.
However, traditional experiments can only monitor the effect of one
factor. Second, while large-scale studies can partially bypass this
issue, participants can hardly stick to their ``restrictive dietary
protocols'', making the results useless. Moreover, there are significant
disagreements between studies sponsored the food industry and those by
independent sources. Results from the former were usually favorable to
their sponsors (Nestle). Therefore, policy makers should require
relevant studies to manifest their relationship with the food industry
in an explicit way. When companies cite these reports, they should also
state the funding sources of these reports.

The last topic is about supersizing, which contributes much to the
obesity epidemic. Wansink and Payne noted that classic recipes contain
much more calories than they did 70 years ago (Wansink and Payne 120 --
121). Soft drinks full of high fructose corn syrup are also sold in
supersize (Surowiecki 123 -- 125). A good measure against this would be
the policy brought by Michael Bloomberg: banning the sale of supersize
drink (Surowiecki 123 -- 124).

\subsection*{Works Cited}
\bibent Pollan, Michael. \textit{The Omnivore's Dilemma: A Natural History of Four Meals}. New York: Penguin, 2006. Print.

\bibent Nestle, Marion. ``Eating Made Simple'' \textit{Scientific American}. Scientific American Inc., 1 Sep. 2007. Web. Rpt. in \textit{Food Matters: A Bedford Spotlight Reader}. Ed. Holly Bauer. Boston: Bedford/St. Martin's, 2014. 72-81. Print.

\bibent Wansink, Brian and Collin R. Payne. ``The Joy of Cooking Too Much: 70 Years of Calorie Increases in Classic Recipes.'' \textit{Annals of Internal Medicine} 50.4 (2009): 291-292. Web. Rpt. in \textit{Food Matters: A Bedford Spotlight Reader}. Ed. Holly Bauer. Boston: Bedford/St. Martin’s, 2014. 119-121. Print.

\bibent Surowiecki, James. ``Downsizing Supersize.'' \textit{The New Yorker}. Conde Nast, 13 Aug. 2012. Web. Rpt. in \textit{Food Matters: A Bedford Spotlight Reader}. Ed. Holly Bauer. Boston: Bedford/St. Martin’s, 2014. 123-125. Print.

\end{mla}
\end{document}
