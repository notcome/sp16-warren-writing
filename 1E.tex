\documentclass[12pt,letterpaper]{article}

\usepackage{ifpdf}
\usepackage{mla}

\begin{document}
\begin{mla}{Minsheng}{Liu}{Cami Koepke}{WCWP 10B}{04/12/16}
  {Industrial Agriculture is More Sustainable}

Are technology advancement and industrialization in the agriculture
sector beneficial? Some people have a negative answer towards this
question. In his book \emph{The Omnivore's Dilemma}, Pollan discusses
the dark side of the highly industrialized agriculture in today's
America. In particular, modern agriculture, according to Pollan, is a
primary culprit of environment pollution. However, there are two issues
Pollan and other supporters of the unindustrialized organic agriculture
fail to see. First, people eat far more meats than they did half an
century ago, and it requires an insane amount of plants to grow so much
meats, which Pollan's ``beyond organic'' agriculture cannot deliver.
Second, from an economic perspective, large-scale production is
inherently more efficient. By transferring the role of farming from
traditional farmers to highly trained experts in food factories, it is
much easier to control pollution and reduce energy waste---plus
necessary governmental supervision is guaranteed. Therefore, the issues
of industrialized agriculture presented in Pollan's book is solvable and
Pollan's proposal has an insuperable problem, which leads to my claim
that it is industrial agriculture that really is the future of
sustainable food production.

Pollan points out that the industrial food system---especially the
conventional agriculture---is wasteful, but he fails to evaluate this
issue comprehensively. In \emph{The Omnivore's Dilemma}, Pollan
criticizes the industrial food system for its burden on our environment.
In particular, Pollan talks about how farmers abuse chemical fertilizers
and the environment crisis brought by excess nitrogen (Pollan 46 -- 47).
However, such pollution is caused by untrained farmers rather than the
synthetic fertilizer \emph{per se}. If farmers scientifically measure
the nitrogen needed, they can use the exact amount of fertilizer
required so that not only the fertilizer is saved---hence saving the
petroleum to produce synthesize fertilizer (Pollan 45)---but there will
also be no pollution caused by the excess nitrogen fertilizer. In fact,
as Paarlberg mentions, such high-tech farming has already become actual
in rich countries for many years (Paarlberg 245). On the other hand,
small-scale organic has its own issue. As pointed out by Biello, a
supporter of organic food, farmers need to learn the knowledge of
maintaining an entire ecosystem, while in conventional agriculture, all
parameters farmers need to know are merely ``synthetic fertilizer,
chemical pesticides, and the like'' (Biello 233). Average American
farmers clearly do not have such expertise, otherwise there will be no
abuse of fertilizers in the first place. Besides, even if Pollan and his
friends successfully persuade the American government to enforce every
farmer to receive education on organic farming techniques, farmers in
the most countries on this planet---poor African and Asian countries in
particular---can never receive such advanced education. The only
feasible solution is to adopt the industrial agriculture and let highly
trained experts to grow food. Using high-tech sustainable conventional
agriculture whenever possible, and when the marginal gain of the organic
agriculture is required, using \emph{industrial} organic agriculture.
For developing countries, foreign direct investment in the agriculture
sector can solve their food crisis.

Beyond the production itself, Pollan also blames the modern food
procession system. Though emotionally understandable, such criticism
makes no sense after one thinks about it rationally. In his description
of Earthbound Farm, a company that provides ready to eat organic salad,
Pollan mentions that it takes ``57 calories of fossil fuel energy for
every calorie of food'', a figure only four percent lower than
conventionally grown salad (Pollan 167). However, such a number cannot
be used to justify Pollan's ``beyond organic'' dream. In his interview
of Salatin, a farmer from the green Polyface Farm, the latter mentioned
that they refused to ship foods because they were ``really serious about
energy''. Customers have to ``drive down here to Swoope to pick it up''
(Pollan 133). Obviously, it spends more energy than delivering a truck
of foods to a market than asking each one to drive to the farm to pick a
small amount of amount. Alternatively, if this ``beyond organic''
paradigm becomes the dominant way to provide food in United States, such
``waste'' of energy is unavoidable. Some may argue that maybe it is the
modern lifestyle we should give up. Instead of eating processed foods,
one should cook by themselves or even grow the food by themselves.
However, such a view neglects the time cost imposed on every member of
our society. Human beings choose to reduce their time spent on eating
foods because in this way they can be happier and create more value.
People are interested in saving energy because they want to protect the
environment, which in turn makes people happier. If protecting the
environment seriously impedes our happiness yet choosing not to do it
has no direct impact on our life, no one will care the environment.
Since it is inevitable to spend a significant amount of energy in
processing and storing foods, the only thing left consider is to choose
a way that minimizes the energy consumption. Due to the efficiency
brought by the large-scale production, industrial food system has a
significant advantage over its old-fashioned competitor. For example,
Earthbound developed their own lettuce-shaving machine (Pollan 166), but
one cannot expect a traditional farmer to develop its own machine.
Besides, such a machine might be so expensive that unless multiple small
farms share one machine the cost will not be worth the investment. The
efficiency in large-scale production was what enabled the prosperous
industrial revolution, there is no reason to believe that the same
factor will fail the food sector.

Last but not least, the ``beyond organic'' farming practice Pollan
boasts simply cannot feed the population. It is a fact that nowadays
meat such as beef and chicken has become a major constituent of an
American's daily meal. In its nutritional guidelines, USDA also
recommends the intake of meat, poultry, fish, and eggs (USDA 113). The
world needs so much meat. However, animals are extremely inefficient in
converting energy stored in plants to energy stored in proteins. To grow
to be slaughtered, a steer has to consume ``the equivalent thirty-five
gallons of oil'' (Pollan 84). While Pollan think Americans grow too many
corns (Pollan 61 -- 64), which reduces the biodiversity (Pollan 38 --
40), the reality is that without those corns there will not be enough
meat to eat. Paarlberg also mentions that if Europe decides to use
organic method to grow livestocks, ``an additional 28 million hectares
of cropland'' are required, which might lead to mass deforestation
(Paarlberg 245). Moreover, it is unreasonable to ask people to stay away
from meat---some might follow and become vegetarian, but the majority
people simply do not care. Hence, it is clear that even though ``beyond
organic'' might be able to feed a margin of our population who pay much
attention to their diets, an industrial food system is indispensable to
keep our society to work.

Overall, issues of the food industry can be surmounted and the idea of
organic agriculture by small farms cannot feed enough people. Indeed,
our lifestyle cause a huge amount of food waste, but that is a price
people from a modern society choose to pay. If someone is interested in
protecting our environment and caring our food health, he or she should
probably work hard to push legislation, to resolve issues such as the
unfair economic distribution between farmers and the food industry, and
to remind people of the potential danger behind the abuse of antibiotic.
However, the industrialization of food production \emph{per se} is not a
bad thing. One should never fear something simply because it is
unnatural, for our very existence relies on changing the nature.

\subsection*{Works Cited}
\bibent Pollan, Michael. \textit{The Omnivore's Dilemma: A Natural History of Four Meals}. New York: Penguin, 2006. Print.

\bibent Paarlberg, Robert. ``Attention Whole Foods Shoppers.'' \textit{Foreign Policy}. The FP Group, May/June 2010. Web. Rpt. in \textit{Food Matters: A Bedford Spotlight Reader}. Ed. Holly Bauer. Boston: Bedford/St. Martin’s, 2014. 240-247. Print.

\bibent Biello, David. ``Will Organic Food Fail to Feed the World?'' \textit{Scientific American}. Scientific American Inc., 25 Apr. 2012. Web. Rpt. in \textit{Food Matters: A Bedford Spotlight Reader}. Ed. Holly Bauer. Boston: Bedford/St. Martin's, 2014. 232-235. Print.

\bibent United States Department of Agriculture.
``The Food Plate and Food Pyramid Nutritional Guidelines''
Rpt. in \textit{Food Matters: A Bedford Spotlight Reader}.
Ed. Holly Bauer. Boston: Bedford/St. Martin's, 2014. 112-113. Print.

\end{mla}
\end{document}
