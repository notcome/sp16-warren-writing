\documentclass[12pt,letterpaper]{article}

\usepackage{ifpdf}
\usepackage{mla}

\begin{document}
\begin{mla}{Minsheng}{Liu}{Cami Koepke}{WCWP 10B}{04/15/16}
  {Industrial Agriculture is Sustainable}

Is industrialization in the agriculture sector beneficial? Some people
have a negative answer towards this question. In his book \emph{The
Omnivore's Dilemma}, Pollan discusses the dark side of the highly
industrialized agriculture in today's America. In particular, modern
agriculture, according to Pollan, is a primary culprit of environment
pollution. He proposes that we should switch to beyond organic, which
requires farmers to maintain a delicate ecological balance so that it is
sustainable and pollution-free (citation required). However, there are
three issues Pollan fail to consider. First, large-scale production is
inherently more efficient. By transferring the role of farming from
traditional farmers to highly trained experts in food factories, it is
much easier to control pollution and reduce energy waste. Second, many
types of waste are caused by our lavish lifestyle, yet it is impossible
to relinquish this modern lifestyle. Third, Pollan and his friends fail
to present convincing data to show that beyond organic can deliver
enough foods. Therefore, the issues of industrialized agriculture
presented in Pollan's book is solvable and Pollan's proposal has an
insuperable problem, which leads to my claim that it is industrial
agriculture that is the future of sustainable food production.

Pollan believes that beyond organic is much more friendly to our
environment than its industrial counterparts, namely conventional
agriculture and industrialized organic agriculture. In practice,
however, such a claim does not hold for two reasons. First, modern
technology can eliminate most pollution caused by industrial
agriculture. For example, Pollan talks about how farmers abuse chemical
fertilizers to ensure that the soil can grow enough corns. This practice
has a severe negative impact on our environment: excess nitrogen will
flow off and concentrate in the sea, which has created a dead zone in
Gulf of Mexico (Pollan 46 -- 47). Nevertheless, this kind of pollution
is caused by untrained farmers rather than the synthetic fertilizer
\emph{per se}. If farmers scientifically measure the nitrogen needed,
they can use the exact amount of fertilizer required and there will be
no excess nitrogen to cause pollution. In fact, as Paarlberg mentions,
such high-tech farming has already become actual in rich countries for
many years (Paarlberg 245).

Another reason that beyond organic cannot beat its industrial
counterparts is that the expertise required for beyond organic cannot be
reproduced in a large scale. While in conventional agriculture all
parameters farmers need to know are merely ``synthetic fertilizer,
chemical pesticides, and the like'' (Biello 233), in organic agriculture
farmers need to maintain an entire ecosystem. As Biello put in his
article, organic agriculture is a ``knowledge-intensive'' sector (Biello
233). The success of capitalism has taught us that it is better to
separate the estate---in this case, farms---from technicians, which is
exactly what industrial agriculture is. After all, average American
farmers are not well-educated in farming, otherwise they would not abuse
synthetic fertilizers in the first place. Besides, if one considers this
issue globally, it is unrealistic to educate farmers in most
countries---poor African and Asian countries in particular---to go
beyond organic. In contrast, it is much wiser to let international
corporations to grow food efficiently. In conclusion, even when organic
agriculture's marginal benefit on environment protection is a necessity,
it is more realistic to employ the industrial model of organic farming.

Beyond production, Pollan also blames the modern food procession system.
This criticism, however, is also untenable. Pollan points out that to
make ready-to-eat organic salad, it takes ``57 calories of fossil fuel
energy for every calorie of food.'' A large amount of energy is
``wasted'' in the large refrigerator in which workers process vegetables
and turn them into salad (Pollan 167). How does beyond organic perform
in this area? Ironically, Pollan eschews this problem completely. In his
interview of Salatin, a beyond organic farmer, the latter mentions that
they refuse to ship foods because they were ``really serious about
energy.'' Customers will have to ``drive down here'' to ``pick it up''
(Pollan133). Since there is no food process nor shipment in beyond
organic, there is of course no energy wasted in this step. Yet one
should take into account the energy each customer spends driving down to
the farm, which is much more than that required to deliver an equivalent
amount of food by a single truck to a supermarket. Beyond organic is by
no means energy-friendly. Energy still is spent, albeit invisibly.

Perhaps it is our lavish lifestyle that Pollan takes as target: instead
of eating highly processed foods, one should prepare their foods by raw
materials and grow these materials if necessary. However, Pollan
neglects the time cost imposed by such a natural lifestyle, which makes
the later impractical. Compared with our immediate interests in saving
more time for our life, our looming concerns over environmental problems
are of no consequence. Choosing not to protect the environment has no
direct impact on our life---it is fishes that are dying in the Gulf of
Mexico, not \emph{Homo sapiens}. Therefore, if saving our environment
requires us to give up time, most people will not follow.

Since it is inevitable to spend energy in processing and storing foods,
the only thing to consider is to choose a way that minimizes the energy
consumption. Contrary to many people's stereotypes, this is a field
where our industrial agriculture can easily beat its naturalism
competitor. The reason is that large-scale production, from an economic
perspective, is inherently more efficient. For example, Earthbound Farm
developed their own lettuce-shaving machine (Pollan 166), which could
save human labor significantly. One cannot expect a traditional farmer
to develop their own machine. Moreover, such a machine might be so
expensive that unless multiple small farms share one machine the cost
will not be worth the investment. The example above---asking each
customer to drive down to the farm versus delivering a batch of food
using a single truck to a supermarket---also exemplifies this point. The
efficiency in large-scale production enabled the prosperous industrial
revolution. There is no reason to believe that the same factor will fail
the food sector.

Last but not least, it is questionable whether beyond organic is capable
of feeding people worldwide with our current demands on foods. Boasters
of organic farming present evidence that we are producing more food than
we need and manage to persuade us that it is food waste that should be
fixed. They seem to ignore two important factors though. First, it is
unclear whether it is easy to eliminate food waste in developed
countries like United States. A significant portion of foods are thrown
away because of our modern lifestyle. For instance, one may clear up
one's refrigerator because labels on foods say that they are expired,
even though these foods are still edible (citation from the video
required, ask for permission). However, as argued above, it is hard to
relinquish our modern lifestyle. People do not have enough time to care
whether foods marked expired in their refrigerators are edible or not.
With economy consistently grows up, it is conceivable that people will
become more and more lavish in consuming foods, putting even higher
demands on food production.

Structural changes in our diets are also under appreciated. Nowadays,
meat and other foods from animals have become major constituents of
people's daily meal. In its nutritional guidelines, USDA also recommends
the intake of meat, poultry, fish, and eggs (USDA 113). This planet
needs meat. However, animals are extremely inefficient in converting
energy stored in plants to proteins. For example, a steer has to consume
``the equivalent thirty-five gallons of oil'' before it is ready for
slaughter (Pollan 84). While some may think that Americans' industrial
food system grow too many corns each year (Pollan 61 -- 64), the reality
is that without those corns there will be a dearth of meat. Paarlberg
also mentions that if Europe decides to use organic method to grow
livestocks, ``an additional 28 million hectares of cropland'' are
required (Paarlberg 245). It is unreasonable to ask everyone to stay
away from meat, and as average life standard keeps improving, the demand
on animal products will only increase. Hence, it is clear that
supporters of the natural farming paradigm need data that are more
convincing to show that beyond organic can indeed replace our
well-established industrial food system without lowering our life
standards significantly.

It is understandable for people to fear new technologies. Pioneers often
pay high prices for their boldness, sharing the fate of those dead
fishes in the Gulf of Mexico. However, industrialization in the
agriculture sector has matured. The polluting image of conventional
agriculture is the past. The advantages brought by large-scale
production have outweighed even the most delicate natural system.
Indeed, there are many issues in American agriculture---like the unfair
distribution of economic gains between small farmers and large
corporations (Pollan 53)---but we should move forward to push
legislation to maximize the potential of industrialization rather than
moving backward to surrender to the nature. The future of sustainable
food production should be using high-tech sustainable conventional
agriculture whenever possible, and when the marginal gain of the organic
agriculture is required, using industrial organic agriculture.


\subsection*{Works Cited}
\bibent Pollan, Michael. \textit{The Omnivore's Dilemma: A Natural History of Four Meals}. New York: Penguin, 2006. Print.

\bibent Paarlberg, Robert. ``Attention Whole Foods Shoppers.'' \textit{Foreign Policy}. The FP Group, May/June 2010. Web. Rpt. in \textit{Food Matters: A Bedford Spotlight Reader}. Ed. Holly Bauer. Boston: Bedford/St. Martin’s, 2014. 240-247. Print.

\bibent Biello, David. ``Will Organic Food Fail to Feed the World?'' \textit{Scientific American}. Scientific American Inc., 25 Apr. 2012. Web. Rpt. in \textit{Food Matters: A Bedford Spotlight Reader}. Ed. Holly Bauer. Boston: Bedford/St. Martin's, 2014. 232-235. Print.

\bibent United States Department of Agriculture.
``The Food Plate and Food Pyramid Nutritional Guidelines''
Rpt. in \textit{Food Matters: A Bedford Spotlight Reader}.
Ed. Holly Bauer. Boston: Bedford/St. Martin's, 2014. 112-113. Print.

\end{mla}
\end{document}
