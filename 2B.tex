\documentclass[12pt,letterpaper]{article}

\usepackage{ifpdf}
\usepackage{mla}

\begin{document}
\begin{mla}{Minsheng}{Liu}{Cami Koepke}{WCWP 10B}{04/27/16}{2B}

From an overarching perspective, Pollan believes what is natural is the
most suitable. The reason behind such a view is simple: what is
currently present is the most stable state of our complex natural
system. Here I give two instances of such an idealism. First, Pollan
points out that Americans should place their faith in some dietary
traditions rather than the so-called ``scientific evidences'' (Pollan
296, 299 -- 303). Two, Pollan presents a rebut for the animal rightists
that prevent killing animals is not necessarily good for a species
(Pollan 322 -- 325).

As for the conversation of food health, only the first example I give
above fits in. In Freedman's essay, the latter argues that it is
impractical to let the largest victim group---namely the poor class---to
switch to a diet consisting mainly of unprocessed foods. Freedman
proposes that the only viable solution is to rely on food giants to
solve the problems for the public good. If one calls Freedman's approach
``top-down'', namely that Freedman relies on the food provider to change
the status quo, Pollan's claim is clearly more ``bottom-up''. Pollan
thinks that we should resort to some rules about what to eat (Pollan
296) so that every food consumers---namely all human beings---are
responsible for what they are eating. Despite the fact that Pollan did
not discuss at all the practical possibility for such ``bottom-up''
approach, I believe the following issues are not well-addressed.

It seems that Pollan takes for granted that traditional cuisines are
good. As Freedman puts, this is more like a religious belief rather than
a serious argument. There is no clear scientific study Pollan draws
from. Since eating too much as an ordinary American citizen does is a
bad practice, merely being superior to such a practice cannot prove
itself good in an absolute sense. Beyond that, one should also note the
fact that cuisines are evolved over the last few thousands as human
society evolved. What has changed in the last century, however, was so
abrupt that one cannot say for sure if our traditional dietary habit did
catch up. There might exists issues that were not exposed but are
starting to manifest as the average life expectance grows.

While Pollan could refute by saying that my argument relies too much on
the idea of a scientifically maximized dietary habit, which is something
he tries to avoid at his best (Pollan 301 -- 303), I would respond by
saying that such a change is inevitable. America is a immigrant country.
Each immigrant group brings their own cuisines. However, as Pollan
himself points out, these cuisines are eased out and replaced by two
dominant approach: the healthy lifestyle in the more affluent class and
the unhealthy intake of junk food in the less affluent class. In other
words, the traditional dietary rituals that Pollan boasts cannot handle
this new environment. One cannot sit there, do nothing, and hope that
some omnipotent guy descends and gives American people a new cuisine
that survives among these food giants. We are in the middle of a radical
change as well as a formation of a new cuisine. Food science has made
substantial progress over the last a few decades. We can now identify
the main culprits of this obesity epidemic. Food companies can make some
real moves to alleviate these problems, maybe solving them completely.
It makes no sense, in such a time, to go back to our traditional
cuisines. Their very disappearing proves their failure. Plus the fact
that food companies and the variety of foods brought by them will not
disappear by themselves, it is clear that such a reactionary view cannot
persist.

Therefore, I believe that Pollan, though making some good contributions
to the conversation, is refuted in a rigors way by Freedman's essay. His
argument is not compelling, and will serve a classic example of a failed
argument.


\subsection*{Works Cited}
\bibent Pollan, Michael. \textit{The Omnivore's Dilemma: A Natural History of Four Meals}. New York: Penguin, 2006. Print.

\end{mla}
\end{document}
