\documentclass[12pt,letterpaper]{article}

\usepackage{ifpdf}
\usepackage{mla}

\begin{document}
\begin{mla}{Minsheng}{Liu}{Cami Koepke}{WCWP 10B}{04/15/16}
  {Industrial Agriculture is Sustainable}

Is industrialization in the agriculture sector beneficial? Some people
have negative answers to this question. In his book \emph{The Omnivore's
Dilemma}, Michael Pollan discusses the dark side of the highly
industrialized agriculture in today's America. In particular, modern
agriculture, according to Pollan, is a primary culprit of environmental
pollution. He classifies our food production pathways into two types:
the industrial way and what he calls ``beyond organic.'' The industrial
agriculture can be further divided into the conventional agriculture and
the industrialized organic agriculture. ``Beyond organic,'' in contrast,
is a form of local farming, which manages to create a self-sustainable
ecosystem such that the whole system requires nothing but solar energy
and leaves no pollution. Pollan argues that we should abandon the
industrial agriculture---be it conventional or organic---and switch to
``beyond organic.'' However, from a practical perspective, there are
three issues Pollan fail to consider. First, the pollution caused by
conventional agriculture is the fault of the execution, not the
technology itself. This environmental issue is treatable and has been
mitigated in the past few decades. Second, a large-scale system is
inherently more efficient. Therefore, when one considers the lifetime of
food products---from farming to processing---and thinks of~the planet as
a whole, one will find that industrial agriculture is greener than
``beyond organic.'' Lastly, the expertise required by ``beyond organic''
limits its potential in being the dominant ``food provider.'' These
three reasons lead to my claim that industrial agriculture is the only
practical solution to our food production currently available.

Pollan believes that ``beyond organic'' is much more friendly to our
environment than the industrialized conventional agriculture. As briefly
introduced above, in ``beyond organic,'' farmers maintain a
self-sustained ecological cycle. Though theoretically attractive, in
practice, ``beyond organic'' has little advantage over modern
conventional agriculture. Pollution from conventional farming is caused
by unscientific practices rather than the technology. For example,
Pollan talks about how farmers abuse chemical fertilizers to ensure that
the soil can grow enough maize. This practice has a severely negative
impact on our environment: excess nitrogen will flow off and concentrate
in the sea, which has created a dead zone in the Gulf of Mexico (Pollan
46 -- 47). Nevertheless, the issue of excess nitrogen is caused by
untrained farmers rather than the synthetic fertilizer \emph{per se}. If
farmers can measure the nitrogen demand scientifically, they will use
the exact amount of fertilizer required. Consequently, there will be no
excess nitrogen to cause pollution. In fact, modern conventional
agriculture has already been using infrared sensors to measure the
amount of nitrogen required, among many other techniques that reduce the
``environmental footprint of modern agriculture'' (Paarlberg 245).
Therefore, it is unnecessary to relinquish the industrialized
method~during the growing stage for environmental concerns.

Another distinctive feature of ``beyond organic'' is that its~farmers do
not process foods as workers of industrial agriculture do. It keeps food
fresh not by refrigerator and additives, but by staying local. Pollan
gives an interesting example. When he interviewed the ``beyond organic''
farmer Joel Salatin, the latter pointed out that they refused to ship
foods because they were ``really serious about energy''; customers would
have to ``drive down here'' to ``pick it up'' (Pollan 133). In contrast,
industrial foods are highly processed to ensure that~they are still
fresh when customers receive them. The whole process, however, is
criticized by Pollan as wasteful. In his description of Earthbound Farm,
an industrial organic company that sells ready-to-eat organic salad,
Pollan points out that it takes ``57 calories of fossil fuel energy for
every calorie of food.'' A significant amount of energy is ``wasted'' in
the large refrigerator in which workers process vegetables and turn them
into salads (Pollan 167). Pollan's attitude is again articulated in his
comparison between the industrial food system---be it conventional
farming or ``big {[}industrial{]} organic''---and ``beyond organic.''
The former is unnatural~and wasteful, and it should be replaced by the
latter.

From a practical view, however, it's impossible to realize Pollan's
ideal. To achieve what he asks, we need to change our lavish dietary
habits. Instead of eating highly processed foods that are designed to be
consumed anywhere, at any time, and in a convenient way, one needs to
dedicate a significant amount of time to pick up fresh foods in small
local markets and cook these foods in a natural and healthy style. Yet
Pollan neglects the time cost imposed by such a lifestyle, which makes
the latter impractical. A new lifestyle cannot become mainstream unless
it provides enough incentives for the general public. Does Pollan's
proposal have sufficient motivations? Hardly enough. Current studies do
not manifest any significant health benefits from eating organic foods
(Nestle 75 -- 76). Also, pollution caused by food production~is a
looming issue---after all, it is fish that are dying in the Gulf of
Mexico, not \emph{Homo sapiens}. Therefore, it is safe to conclude that
the majority of us will not give up our dietary habits based~on the
current arguments presented by Pollan.

Next, we can examine why, with our current lifestyle, industrial
agriculture is more energy-saving than ``beyond organic.'' The key lies
in the inherent efficiency brought by large-scale production. Consider
the example of food delivery discussed above. If ``beyond organic''
becomes dominant, each customer will~need to drive down to the farm,
and~the energy spent during this~travel is much greater than that
required to deliver an equivalent amount of food by a single truck to a
local market. ``Beyond organic'' is by no means energy-friendly. Energy
is~still spent, albeit invisibly. Moreover, when the yield is large
enough, foods can be delivered by trains, like what Americans are doing
to industrial corns (Pollan 61). Even cargo ships can be used to
transport food across oceans, which enables us to grow foods in more
fertile lands and deliver to barren areas, reducing the use of
fertilizers, or to~deliver to desert regions, decreasing the energy
spent on irrigation. Both types~of~transportation are more efficient
than cars in terms of energy consumption. While Pollan blames
Earthbound's large refrigerator, one should not forget that, when the
market demands ``beyond organic'' farms to provide similar ready-to-eat
products, these farms will need to build their own small refrigerators.
These small refrigerators will consume more energy than a single large
refrigerator, or Earthbound would~not~have~built such a giant machine in
the first place. Maybe there will be a company dedicated to making such
products, owning a giant refrigerator, and collecting materials from
``beyond organic'' farms, but what then is the advantage ``beyond
organic'' has over its industrial counterpart? When people talk about
industrialization, the words ``efficiency'' and ``pollution'' often come
into mind simultaneously. However, they forget that the very principle
that helps reduce monetary costs can also be used to reduce
environmental costs. Consequently, only the industrial food system is
\emph{practically} friendly to our environment.

Last but not least, the expertise required for ``beyond organic'' cannot
be reproduced on a large scale, which disqualifies ``beyond organic'' as
a potential food provider. While in conventional agriculture, all
the~parameters farmers need to know are merely ``synthetic fertilizer,
chemical pesticides, and the like'' (Biello 233), in organic farming,
farmers need to maintain an entire ecosystem. As Biello put in his
article, organic agriculture is a ``knowledge-intensive'' sector (Biello
233). The success of capitalism has taught us that it is better to
separate the estate---in this case, farms---from technicians, which is
exactly what industrial agriculture does. After all, average American
farmers are not well-educated in farming---otherwise, they would not
abuse synthetic fertilizers in the first place. Besides, if one
considers this issue globally, it is unrealistic to educate farmers in
most countries---poor African and Asian countries in particular---to go
``beyond organic.'' In contrast, it is much wiser to let international
corporations grow food efficiently. In conclusion, even when organic
agriculture's marginal benefit on environmental protection is a
necessity, it is more realistic to employ the industrial model of
organic farming.

It is understandable for people to fear new technologies. Pioneers often
pay high prices for their courage, sharing the fate of those dead fish
in the Gulf of Mexico. However, industrialization in the agricultural
sector has matured. The polluting image of conventional agriculture is
the past. The advantages brought by large-scale production have
outweighed even the most delicate natural system. Moreover, ``beyond
organic'' cannot be reproduced on a large~scale to make it a real
competitor for the industrial food system. Indeed, there are many issues
in American agriculture---like the unfair distribution of economic gains
between small farmers and large corporations (Pollan 53)---but
we~should~be~pushing legislation to maximize the potential of
industrialization rather than moving backward to surrender to nature.
The future of sustainable food production should be using high-tech,
sustainable, conventional agriculture whenever possible, and when the
marginal gain of organic farming is necessary, we~should~be~using
\emph{industrial} organic farming.

\subsection*{Works Cited}
\bibent Biello, David. ``Will Organic Food Fail to Feed the World?''
\textit{Scientific American}. Scientific American Inc., 25 Apr. 2012. Web. Rpt.
in \textit{Food Matters: A Bedford Spotlight Reader}. Ed. Holly Bauer. Boston:
Bedford/St. Martin's, 2014. 232-235. Print.

\bibent Nestle, Marion. ``Eating Made Simple'' \textit{Scientific American}.
Scientific American Inc., 1 Sep. 2007. Web. Rpt. in \textit{Food Matters: A
Bedford Spotlight Reader}. Ed. Holly Bauer. Boston: Bedford/St. Martin's, 2014.
72-81. Print.

\bibent Paarlberg, Robert. ``Attention Whole Foods Shoppers.'' \textit{Foreign
Policy}. The FP Group, May/June 2010. Web. Rpt. in \textit{Food Matters: A
Bedford Spotlight Reader}. Ed. Holly Bauer. Boston: Bedford/St. Martin’s, 2014.
240-247. Print.

\bibent Pollan, Michael. \textit{The Omnivore's Dilemma: A Natural History of
Four Meals}. New York: Penguin, 2006. Print.
\end{mla}
\end{document}
